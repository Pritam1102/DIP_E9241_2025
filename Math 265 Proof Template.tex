\documentclass[12 pt]{article}        	%sets the font to 12 pt and says this is an article (as opposed to book or other documents)
\usepackage{amsfonts, amssymb, amsmath}					% packages to get the fonts, symbols used in most math

%\usepackage{setspace} % Together with \doublespacing below allows for doublespacing of the document

\oddsidemargin=-0.5cm                 	% These three commands create the margins required for class
\setlength{\textwidth}{6.5in}         	%
\addtolength{\voffset}{-20pt}        		%
\addtolength{\headsep}{25pt}           	%



\pagestyle{myheadings}                           	% tells LaTeX to allow you to enter information in the heading
\markright{Pritam Kumar\hfill \today \hfill} 	


\newcommand{\eqn}[0]{\begin{array}{rcl}}%begin an aligned equation - allows for aligning = or inequalities.  Always use with $$ $$
\newcommand{\eqnend}[0]{\end{array} }  	%end the aligned equation

\newcommand{\qed}[0]{$\square$}        	% make an unfilled square the default for ending a proof

%\doublespacing                         	% Together with the package setspace above allows for doublespacing of the document

\begin{document}												% end of preamble and beginning of text that will be printed

\section{Spatial Smoothing Filters}
Spatial smoothing filters are a concept in image processing used to reduce noise 
and smooth out variations in an image by averaging or combining pixel values 
in a local neighborhood.

\vspace{4pt}
\textbf{Types of Filters:}
\begin{enumerate}
    \item \textbf{Averaging Filter:} Replaces each pixel with the average of its neighbours.  
    \[
    \omega_1 = \frac{1}{9}
    \begin{bmatrix}
        1 & 1 & 1 \\
        1 & 1 & 1 \\
        1 & 1 & 1
    \end{bmatrix}
    \]

    \item \textbf{Gaussian Filter:} Weights the neighbors according to a Gaussian distribution, 
    giving higher weight to pixels closer to the center.  
    \[
    h(m,n) = K\, e^{-(m^2+n^2)}, \qquad m,n \in \{-1,0,1\}.
    \]
    K is chosen such that 
    \[
    \sum_{m,n}{h(m,n} = 1
    \]
    or
    \[
K=\left(\sum_{m=-1}^{1}\sum_{n=-1}^{1} e^{-(m^2+n^2)}\right)^{-1}.
\]
    
    \item \textbf{Median Filter:} A nonlinear filter that replaces each pixel 
    with the median of its neighborhood.  
    Useful for removing \emph{salt-and-pepper noise}.

    \item \textbf{Weighted Average Filter:} Similar to the averaging filter, 
    but assigns more importance to the central pixel.  
    Example:  
    \[
    \omega_3 = \frac{1}{10}
    \begin{bmatrix}
        1 & 1 & 1 \\
        1 & 2 & 1 \\
        1 & 1 & 1
    \end{bmatrix}
    \]
\end{enumerate}

\section{Applications of Spatial Smoothing Filters}
Spatial smoothing filters are widely used in image processing tasks. 
Some common applications include:
\begin{enumerate}
    \item \textbf{Denoising:} 
    By averaging neighboring pixels, smoothing filters reduce the impact 
    of random noise, making the image cleaner for further processing.

    \item \textbf{Preprocessing before Binarization:} 
    Smoothing helps suppress small intensity variations so that threshold-based 
    binarization produces cleaner object boundaries.

    \item \textbf{Edge Detection Preprocessing:} 
    Many edge detection algorithms (such as Canny) first apply Gaussian smoothing 
    to reduce noise. This ensures that false edges caused by random pixel fluctuations 
    are minimized.

    \item \textbf{Image Pyramid / Downsampling:} 
    Before reducing image resolution, Gaussian smoothing is applied to avoid aliasing. 
    This is a fundamental step in building image pyramids for multi-scale analysis 
    (e.g., object detection and image compression).
\end{enumerate}

\section{Spatial Sharpening Filters}





\end{document}